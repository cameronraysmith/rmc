A \emph{category} $\mathcal{C}$ consists of the following data:
\begin{enumerate}
\item a class $\operatorname{ob}(\mathcal{C})$ of objects (of $\mathcal{C}$)
\item for each ordered pair $(A,B)$ of objects of $\mathcal{C}$, a collection (we will assume it is
 a set) $\hom(A,B)$ of morphisms from the domain $A$ to the codomain $B$
\item a function $\circ:\hom(A,B)\times\hom(B,C)\to\hom(A,C)$ called composition.
\end{enumerate}

We normally denote $\circ(f,g)$ by $g \circ f$ for morphisms $f,g$. The above data must satisfy the following axioms: for objects $A,B,C,D$,

\textbf{A1}: $\hom(A,B) \cap \hom(C,D)=\emptyset$ whenever $(A,B)\neq (C,D)$, i.e. the intersection is non-trivial only when $A=C$ and $B=D$.

\textbf{A2}: (Associativity) if $f \in \hom(A,B)$, $g\in\hom(B,C)$ and $h\in\hom(C,D)$, $h\circ (g\circ f)=(h\circ g)\circ f$

\textbf{A3}: (Existence of an identity morphism) for each object $A$ there exists an identity morphism $ {}id_{A}\in\hom(A,A)$ such that for every $f\in\hom(A,B)$, $f\circ id_{A}=f$ and $ {}id_{A}\circ g=g$ for every $g \in \hom(B,A)$.

Some examples of categories:
\begin{itemize}
\item \textbf{0} is the empty category with no objects or morphisms, \textbf{1} is the category with one object and one (identity) morphism.
\item If we assume we have a universe $U$ which contains all sets encountered in ``everyday'' mathematics,
\textbf{Set} is the category of all such small sets with morphisms being set functions
\item \textbf{Top} is the category of all small topological spaces with morphisms continuous functions 
\item \textbf{Grp} is the category of all small groups whose morphisms are group homomorphisms 
\end{itemize}

\textbf{Remark}.  If $\hom(A,B)$ in the second condition above is not required to be a set (but a class), we usually call $\mathcal{C}$ a \emph{large category}.